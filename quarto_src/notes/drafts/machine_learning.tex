In the previous sections, we explored perspectives on \emph{unsupervised} machine learning tasks, which aim to understand a data set in isolation without guidance or predefined outcomes.
For instance, we can infer the probability of heads from a sequence of coin flips or deduce the group structure from a network using only the pattern of connections. We also discussed how measures like Bayesian evidence or description length assess the quality of our model in an information-theoretic manner that is intrinsic to the data.

After extracting these inferences, we can leverage and evaluate them beyond the scope of the initial data set. One application is to \emph{predict} future events; for example, in a college football network of match outcomes, we might predict the winner between two teams that did not compete during the regular season. Additionally, we may \emph{validate} our inferences against expert knowledge or existing context, or compare the outputs of different models applied to the same data set. Machine learning offers a variety of tools to address these practical purposes.

If we assume that the same mechanisms that generated our observed data also inform unobserved outcomes, we can leverage our model inferences to make predictions. For example, if we observe a coin and are convinced it is fair, we may predict that future flips of the coin will land heads and tails with equal probability. This extrapolation may or may not be accurate. In machine learning terminology, we initially \emph{fit} the model to the "training" data and then assess the quality of the resulting predictions using a "testing" data set. 

In our coin flip example, we may split the sequence~$\vec{s}$ we observe into a training set~$\vec{s}^{\text{train}}$ of~$n^{\text{train}}$ flips and testing set~$\vec{s}^{\text{test}}$ of~$n^{\text{test}}$ flips. Figure~\ref{fig:coin-flip-perspectives}d provides a schematic of this \emph{cross-validation} set up. After fitting the model to the training data we obtain the posterior distribution of the probability~$p$, represented as $P(p|\vec{s}^{\text{train}})$, which is maximized by the best fit \begin{align}
    \hat{p}_{\text{MAP}}^{\text{train}} = \frac{n_H^{\text{train}} + 20}{n^{\text{train}} + 40}
\end{align} as in Eq.\eqref{eq:coin-flip-p-MAP}. Assuming the withheld testing data~$\vec{s}^{\text{test}}$ is governed by the same parameter~$\hat{p}_{\text{MAP}}^{\text{train}}$ as the \emph{training} data, we can evaluate the likelihood Eq.~\eqref{eq:coin-flip-likelihood} on the \emph{testing} data \begin{align}
P(\vec{s}^{\text{test}}|\hat{p}_{\text{MAP}}^{\text{train}}) = \left(\frac{n_H^{\text{train}} + 20}{n^{\text{train}} + 40}\right)^{n_H^{\text{test}}}\left(\frac{n_T^{\text{train}} + 20}{n^{\text{train}} + 40}\right)^{n_T^{\text{test}}}. \label{eq:log-likelihood-cross-validation}
\end{align}
This serves as a measure of the model's out-of-sample predictive performance. 

While most cross-validation tests use the single best parameter, we can instead use the full posterior distribution of possible parameters to compute the \emph{posterior predictive} \begin{align}
    P(\vec{s}^{\text{test}}|\vec{s}^{\text{train}}) &= \int P(\vec{s}^{\text{test}}|p)P(p|\vec{s}^{\text{train}}) dp \nonumber \\
    &= \frac{(n^{\text{train}} + 41)!(n_H + 20)!(n_T + 20)!}{(n + 41)!(n_H^{\text{train}} + 20)!(n_H^{\text{train}} + 20)!}.
\end{align}
This distribution is equal to the probability that the model generates the data~$\vec{s}^{\text{test}}$ conditioned on it also generating~$\vec{s}^{\text{train}}$.  

In a cross validation context, the initial data set~$\vec{s}$ is randomly split into the training and testing data sets, often at a 80/20 ratio. The predictive performance of the model is quantified using either the likelihood or posterior predictive. In keeping with the information theoretic interpretation Eq.~\eqref{eq:H-log-P-information}, we typically report the negative log likelihood or posterior predictive as \begin{align}
    \langle H(\vec{s}^{\text{test}}|\hat{p}_{\text{MAP}}^{\text{train}})\rangle_{\vec{s}^{\text{test}},\vec{s}^{\text{train}}} &= \langle -\log P(\vec{s}^{\text{test}}|\hat{p}_{\text{MAP}}^{\text{train}})\rangle_{\vec{s}^{\text{test}},\vec{s}^{\text{train}}},
    \nonumber \\
    \langle H(\vec{s}^{\text{test}}|\vec{s}^{\text{train}})\rangle_{\vec{s}^{\text{test}},\vec{s}^{\text{train}}} &= \langle -\log P(\vec{s}^{\text{test}}|\vec{s}^{\text{train}})\rangle_{\vec{s}^{\text{test}},\vec{s}^{\text{train}}}, \label{eq:log-posterior-predictive-cross-validation}
\end{align}
where the results are averaged over many possible validation splits~$\vec{s}^{\text{train}},\vec{s}^{\text{test}}$. In practice the likelihood and posterior predictive can give different results, but we will generally prefer to use the latter to evaluate the full posterior of possible parameter values. 

The Bayesian evidence can also be viewed as a measure of predictive performance, averaged over various data splits. We can write out our data set~$\vec{s}$ as the sequence of coin flips~$s_1,...,s_n$. Bayesian evidence is the probability that the model generates this entire sequence. Meanwhile, the posterior predictive is the probability that the model generates some new piece of data given what it has already generated. By sampling the posterior predictive one coin flip at a time, we can therefore \emph{sequentially} generate the full sequence. 

We start by sampling the first flip~$s_1$, which is equally likely \textit{a priori} to be heads or tails. This outcome informs the next coin flip, drawn according to the posterior predictive~$P(s_2|s_1)$. This repeats until the final coin is predicted using all preceding results using~$P(s_n|s_{n-1},...,s_1)$. By definition of the posterior predictive, the overall probability of generating any given sequence of observations must then equal the Bayesian evidence as
\begin{align}
    P(\vec{s}) &= P(s_n,s_{n-1},...,s_1) \nonumber \\
    &= P(s_n|s_{n-1},...,s_1)...P(s_2|s_1)P(s_1).
\end{align}
From the logarithm of this equation, the description length of the data is the sum over the log-posterior-predictives at each step: \begin{align}
    H(\vec{s}) = H(s_n|s_{n-1},...,s_1) + ... + H(s_2|s_1) + H(s_1).
\end{align}
This relationship holds regardless of the order in which the coin flips are considered. Therefore, the normalized description length is also equal to a suitably defined average \begin{align}
    \frac{1}{n}H(\vec{s}) = \langle H(s_i|\vec{s}^{\text{train}})\rangle_{i,\vec{s}^{\text{train}}}
\end{align}
over all possible subsets of training data and choices of single withheld test point~$s_i$~\cite{FH20}.

We can thus use the Bayesian evidence not only as an information theoretic measure for model selection, but also as an indicator of overall predictive power. However, in keeping with much of the machine learning literature we will often report cross-validation results using the log-likelihood Eq.~\eqref{eq:log-likelihood-cross-validation} and log-posterior-predictive Eq.~\eqref{eq:log-posterior-predictive-cross-validation} in this thesis.

Beyond prediction, we would often like to assess the quality of the inferred parameters directly. If we know from an artificial or empirical context that a parameter truly has a certain value, how does our inferred value compare? One way to establish such a "true" parameter value is in a \emph{synthetic} test where we first draw a true value of the parameter~$p^{\text{true}}$ from the prior~$P(p)$. We then sample an artificial data set~$\vec{s}$ from the model likelihood~$P(\vec{s}|p^{\text{true}})$. Based solely on the resulting data~$\vec{s}$, we then infer the parameter~$p$ and compare it to the underlying~$p^{\text{true}}$. 

In this Bayesian setting, the posterior~$P(p|\vec{s})$ is by definition precisely the distribution of the parameters~$p$ that could have resulted in the observation~$\vec{s}$. Thus, the full posterior distribution gives a complete and optimal description of the truth. Compared to this benchmark, synthetic tests provide valuable test cases to understand deviations in the inferences. For example, we can examine how inferences differ when models are misspecified and do not align with the actual generative process. Understanding this robustness is crucial when applying models to real data, where they very likely do not match the real generative process. 

Even when we consider the posterior of the true model, we may observe how point estimate summaries differ from the true value. Depending on how we quantify the distance between the inference~$\hat{p}$ and the truth~$p^{\text{true}}$, different point estimates may be appropriate. If we define success as only when we get the parameter exactly right (using a "one-hot" metric), we should report the MAP estimate since it maximizes this posterior probability. However, if we aim to minimize the squared error ($\ell_2$ metric) of our inference, we should report the expected a posteriori (EAP) value, which provides the least squares estimate over the posterior. Thus even in the idealized scenario where the data is generated by model, our choice of metric over the parameters influences how we should summarize the inference, either with the mode or the mean of the posterior.  

While we can optimize our point estimates accordingly, the posterior distribution can often be highly dispersed or even multimodal. This means that, given the data, multiple parameter values may fit equally well. The true parameter could reside at any of these peaks, meaning that no single point estimate can reliably be close to the truth. Many inference problems undergo a transition between a noisy regime where it is not possible to consistently identify the generating parameters to a data-rich regime where it becomes feasible. Section~\ref{sec:SBM} discusses such an example in the context of finding group structures in networks, which corresponds to the phase transition of the Ising model at its critical temperature.

In this thesis, we will employ synthetic tests, cross-validation, and parameter metrics to better understand the performance of our network models. Applying these validation frameworks to networks presents unique challenges. For instance, when examining the group structure of a network, we need to evaluate the quality of group identity parameters. Unlike the real probability $p$ of coin flips, there is no inherent notion of "distance" or "mean" among group labelings, which are categorical variables, to facilitate comparison.

In Chapter~\ref{chp:information} we discuss information-theoretic measures to assess the similarity between two such clusterings of the same set of objects. We then apply this measure in synthetic tests to observe the relative performance of commonly used algorithms to recover the ground truth groups used to generate the network. In this picture we also observe regimes or types of group structure where all algorithms struggle to recover the truth. 

The projects considered in this work involve ideas borrowed from the disciplines discussed in all of these Appendices, often in ways that do not cleanly separate into a single category. In Figure~\ref{fig:network-perspectives} we have illustrated schematics of these applications across the thesis.